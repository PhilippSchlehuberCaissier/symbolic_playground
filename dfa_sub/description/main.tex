% Preamble
\documentclass[11pt]{article}

% Packages
\usepackage{amsmath, amssymb}
\usepackage[numbers]{natbib}


% Document
\begin{document}

\section{Signature based minimization of symbolic automata}

In \cite{babiak.13.spin} a signature based algorithm for minimization is presented.
The main idea is to compute a formula (represented as BDD) encoding the
successor classes of a state. That is for which letters (or rather for
which formula over the APs, since we work on semi-symbolic automata) we
go to which class.

If the signatures of two states are identical, they are bisimilar or
language equivalent. This induces equivalence classes which can then
be used to minimize the original automaton.

The original algorithm also took into account the colors
of the transitions, but since we work on DFAs, we can simplify
the algorithm in this regard.
I further adopted the algorithm to work, as much as possible, with the
symbolic representation.

The signatures are recomputed iteratively until a fix point is reached.

At iteration $i$ the signature of states $s$, denoted $sig^i(s)$ is computed as
\begin{equation}
    sig^i(s) = \bigvee_{(s,l,s') \in T} l\ \&\  class^i(s')
\end{equation}
where $class$ returns the current class (which corresponds to the
signature during the last iteration) of the given state encoded as BDD.
At the beginning, only two classes exist, separating accepting and non-accepting
states.

By induction one can show that a fix point will be reached (at latest when
there are as many classes as there are states, in which case the original automaton
was already minimal).

Currently, the recomputation of $class$ is fairly expensive as we have to loop
over all the states (in the symbolic representation this corresponds
to minterms or valuations of the state variables).

By regrouping and ordering all variables in such a way that we have
$(X, AP, X', C, C')$ where $X$/$X'$ are the primed and unprimed state
variables, $AP$ are the automatic input and output propositions of the
automaton and $C$/$C'$ are the propositions used to encode the classes,
we believe that performance can be (considerably) improved.

In this case, looping over the minterms can be replaced by traversing the
BDD which should be significantly more efficient.



\bibliographystyle{abbrvnat}
\bibliography{mybib}

\end{document}